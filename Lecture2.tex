\documentclass{article}
\begin{enviornment}
    
\end{this}

\usepackage{hyperref}
\usepackage{listings}
\usepackage[dvipsnames]{xcolor}
\usepackage{amsmath}

\title{}
\author{Ryan Baker}
\date{\today}

\definecolor{JadeGreen}{RGB}{91,158,62}
\lstdefinestyle{catppuccin}{
	backgroundcolor=\color{White},
	commentstyle=\color{Gray},
	numberstyle=\footnotesize\ttfamily\color{Gray},
	stringstyle=\color{JadeGreen},
	keywordstyle=\color{BurntOrange},
	basicstyle=\ttfamily\footnotesize\color{Black},
	breakatwhitespace=false,
	breaklines=true,
	captionpos=b,
	keepspaces=true,
	numbers=left,
	numbersep=5pt,
	showspaces=false,
	showstringspaces=false,
	showtabs=false,
	tabsize=4,
}
\lstset{style=catppuccin}

\hypersetup{
    colorlinks=true,
    linkcolor=Violet
}

\begin{document}

\maketitle
\tableofcontents
\pagebreak

\subsection*{Lecture Objectives}

\section{Functions}

\begin{itemize}
	\item What is a function?
	\begin{itemize}
		\item In programming, a function is a reusable block of code
		\item It (optionally) takes input and (optionally) returns an output
	\end{itemize}
	\item Why use functions?
	\begin{itemize}
		\item We often want to repeat the same behavior on different pieces of data
		\item Rather than pasting the same code many times, we use a function
		\begin{itemize}
			\item Functions help to keep code maintainable and readable
		\end{itemize}
		\item There is a balance to strike when extracting code into functions
		\begin{itemize}
			\item Too few functions results in long and repetitive code
			\item Too many functions will result in sub-optimal performance and a code base that is very hard to read \begin{itemize}
				\item Every time a function is called a new frame needs to be pushed to the stack and we jump around the executable
			\end{itemize}
		\end{itemize}
	\end{itemize}
	\item How to define a function: \textbf{type name(arguments)}
	\begin{itemize}
		\item \textbf{type}: The return type of the function (can be \texttt{void})
		\item \textbf{name}: The function's name
		\item \textbf{arguments}: The input arguments to a function
		\begin{itemize}
			\item Specified as \textbf{type name} in a comma seperated list
			\item \textbf{Example}: \texttt{int add(int a, int b, int c) \{...\}}
		\end{itemize}
		\item Together, the function's name and arguments make up the signature
	\end{itemize}
	\item How to call a function: \texttt{name(arguments)}
	\begin{itemize}
	\item \textbf{Example}: \texttt{int sum = add(1, 2, 3); // sum = 6}
	\end{itemize}
\end{itemize}

\section{Scope}

\noindent
In C++, \textbf{scope} refers to the region of a program, where a variable or function is accessible and valid. It defines the part of the code where the name of an identifier is recognized.

\begin{itemize}
	\item How is scope defined?
	\begin{itemize}
		\item A scope is defined by curly braces: \texttt{\{...\}}
		\item \texttt{...} is said to be ``within the scope'' and everything else is outside
	\end{itemize}
	\item Types of scope:
	\begin{itemize}
		\item Global scope: Accessible from any part of the program
		\begin{itemize}
			\item Variables declared outside any function
		\end{itemize}
		\item Local scope: Accessible only within the same block (\texttt{\{...\}})
		\begin{itemize}
			\item Variables declared within functions
		\end{itemize}
	\end{itemize}
\end{itemize}

\section{Conditions and Branches}

\subsection{Boolean Statements}

\begin{itemize}
	\item What is a Boolean statement?
	\begin{itemize}
		\item A \texttt{bool} type in C++ can either hold \texttt{true} or \texttt{false}
		\item Therefore, a Boolean statement evaluates to \texttt{true} or \texttt{false}
		\item They are often the inputs to conditional evaluation (for clear reasons)
	\end{itemize}
	\item How to create a Boolean statement:
	\begin{itemize}
		\item \texttt{bool(x)}: casts \texttt{x} to a type \texttt{bool}
		\begin{itemize}
			\item For numeric types, returns \texttt{true} if \texttt{x} $\neq 0$, \texttt{false} if \texttt{x} $ = 0$
			\item \texttt{bool(1) == true;    bool(-1) == true;    bool(0) == false}
		\end{itemize}
		\item Comparison operators: return a boolean based on the comparison
		\begin{itemize}
			\item \texttt{==}, \texttt{!=}, \texttt{<}, \texttt{>}, \texttt{<=}, \texttt{>=} are all rather simple ($\texttt{a == b} \leftrightarrow a=b$)
		\end{itemize}
		\item Unary \texttt{!} operator: negation operator
		\begin{itemize}
			\item \texttt{!true == false $\quad$ !false == true}
			\item \texttt{!(5 == 6) == true $\quad$ !(0 == 0) == false}
		\end{itemize}
		\item Operator \texttt{\&\&}: and operator
		\begin{itemize}
			\item The \textbf{and} operator, \texttt{\&\&} returns true if both sides are true
			\item \texttt{(1 == 1) \&\& (0 == 0) == true}
			\item \texttt{(1 > 0) \&\& (0 > 1) == false}
		\end{itemize}
		\item Operator \texttt{||}: or operator
		\begin{itemize}
			\item The \textbf{or} operator, \texttt{||} returns true if either side is true
			\item \texttt{(1 == 1) || (1 == 0) == true}
			\item \texttt{(1 < 0) || (0 > 1) == false}
		\end{itemize}
	\end{itemize}
\end{itemize}

\subsection{\texttt{if} Statements}

\begin{itemize}
	\item When to use an \texttt{if} statement:
	\begin{itemize}
		\item Use \texttt{if} statements when you want to run a block of code conditionally
		\item \textbf{Example}: You only want to log a user in \textbf{if} their password is correct
	\end{itemize}
	\item \texttt{if} statement syntax: \texttt{if (condition) \{...\}}
	\begin{itemize}
		\item Only executes the block of code (...) if the boolean expression is true
	\end{itemize}
	\item Use of \texttt{else} to provide an alternative
	\begin{itemize}
		\item \texttt{if (condition) \{...\} else \{...\}}
	\end{itemize}
	\item Chaining \texttt{if} and \texttt{else} into \texttt{else if}
	\begin{itemize}
		\item \texttt{if (cond 1) \{...\} else if (cond 2) \{...\} else \{...\}}
		\item \texttt{else if} is not a new keyword, rather an \texttt{if} within an \texttt{else}
	\end{itemize}
	\item The overhead of \texttt{if} statements
	\begin{itemize}
		\item Every time an \texttt{if} statement is evaluated, the computer has to ``jump'' around the executable to the correct execution block
		\item This entails an expensive load operation, making \texttt{if} statements relatively expensive
		\item Overhead is minuscule, why even bother explaining this?
		\begin{itemize}
			\item It's important to understand how things actually work
			\item Builds an understanding for the motivation of \texttt{constexpr}
			\item Sometimes you do actually care about nanoseconds-microseconds
		\end{itemize} 
		\item For these reasons, programmers often use ternary operator
		\begin{itemize}
			\item \texttt{(condition) ? (if true) : (if false)}
			\item ``returns'' second argument if first argument (boolean) is true, else ``returns'' third argument
		\end{itemize}
		\item There are also often mathematical ways to get around using \texttt{if}
	\end{itemize}
\end{itemize}

\subsection{\texttt{switch} Statements}

\begin{itemize}
	\item Think of \texttt{switch}es as \texttt{if}s where you have an \texttt{int} condition, not a \texttt{bool}
	\item \texttt{switch} statement syntax: \texttt{switch(var) \{ cases... \}}
	\begin{itemize}
		\item The case corresponding to the \texttt{int} expression will be executed
		\item If no case matches, \texttt{default} is executed
	\end{itemize}
	\item The \texttt{break} statements are needed, else every case below will also execute
	\begin{itemize} \item Sometimes this is desired behavior, but rarely \end{itemize}
	\item Switches are often preferred to chained \texttt{if}-\texttt{else} statements when possible
	\begin{itemize}
		\item They do not have the same overhead as many \texttt{if}s
	\end{itemize}
\end{itemize}

\section{Loops}

\noindent
Often times, we need a certain block of code to run multiple times. Rather than copying and pasting said block, we can use a loop. A loop is a structure that allows a block of code to be run a specified amount of times.

\subsection{\texttt{while} Loops}

\begin{itemize}
	\item When to use a \texttt{while} loop?
	\begin{itemize}
		\item You have a block that you need to run \textbf{while} a condition is true
		\item \textbf{Example}: \textbf{while} the player is playing the game, render the screen
	\end{itemize}
	\item \texttt{while} loop syntax: \texttt{while (condition) \{...\}}
	\begin{itemize}
		\item \texttt{...} will run \texttt{while} condition is true
	\end{itemize}
	\item When to use a \texttt{do while} loop?
	\begin{itemize}
		\item You need the block of code to run at least one time
		\item \texttt{do \{...\} while (condition)}
		\begin{itemize}\item The \texttt{do} keyword's only purpose is to pair the \texttt{while} loop to a scope written before it, rather than after
		\end{itemize}
		\item A \texttt{do while} loop runs the block first, and checks the condition second
	\end{itemize}
\end{itemize}

\subsection{\texttt{for} Loops}

\begin{itemize}
	\item When to use a \texttt{for} loop?
	\begin{itemize}
		\item You need a block of code to run a set number of times
		\begin{itemize}
			\item \texttt{for} loops are much more flexible than \texttt{while} loops
			\item Any \texttt{while} loop can be a \texttt{for} loop
		\end{itemize} 
	\end{itemize}
	\item \texttt{for} loop syntax: \texttt{for (init; condition; update) \{...\}} 
	\begin{itemize}
		\item \textbf{init} is a piece of code that will run once before the loop runs
		\item \textbf{condition} works just like a \texttt{while} loop
		\item \textbf{update} runs at the end of every loop iteration
		\item You can leave any or all fields empty \begin{itemize}
			\item The default behavior for an empty condition is ``\texttt{true}''
		\end{itemize}
	\end{itemize}
\end{itemize}

\subsection{Control Flow Statements}

\subsubsection{\texttt{break}}

\begin{itemize}
	\item \texttt{break} is used to ``break out'' of loops and switches early
	\item Can only be used within a loop or a switch, not any scope
\end{itemize}

\subsubsection{\texttt{continue}}

\begin{itemize}
	\item \texttt{continue} is used to ``continue'' to the next iteration of a loop
	\item Can only be used within a loop structure
\end{itemize}

\subsubsection{\texttt{return}}

\begin{itemize}
	\item \texttt{return} is used to ``return'' a value from a function
\end{itemize}

\end{document}
