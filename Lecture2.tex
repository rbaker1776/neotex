\documentclass{article}

\usepackage{hyperref}
\usepackage{listings}
\usepackage[dvipsnames]{xcolor}
\usepackage{amsmath}

\title{}
\author{Ryan Baker}
\date{\today}

\definecolor{JadeGreen}{RGB}{91,158,62}
\lstdefinestyle{catppuccin}{
	backgroundcolor=\color{White},
	commentstyle=\color{Gray},
	numberstyle=\footnotesize\ttfamily\color{Gray},
	stringstyle=\color{JadeGreen},
	keywordstyle=\color{BurntOrange},
	basicstyle=\ttfamily\footnotesize\color{Black},
	breakatwhitespace=false,
	breaklines=true,
	caption=b,
	keepspaces=true,
	numbers=left,
	numbersep=5pt,
	showspaces=false,
	showstringspaces=false,
	showtabs=false,
	tabsize=4,
}
\lstset{style=catppuccin}

\hypersetup{
    colorlinks=true,
    linkcolor=B,
}

\begin{document}

\maketitle
\tableofcontents
\pagebreak

\subsection*{Lecture Objectives}

\section{Functions}

\begin{itemize}
	\item What is a function?
	\begin{itemize}
		\item In programming, a function is a reusable block of code
		\item It (optionally) takes input and (optionally) returns an output
	\end{itemize}
	\item Why use functions?
	\begin{itemize}
		\item We often want to repeat the same behavior on different pieces of data
		\item Rather than pasting the same code many times, we use a function
		\begin{itemize}
			\item Functions help to keep code maintainable and readable
		\end{itemize}
		\item There is a balance to strike when extracting code into functions
		\begin{itemize}
			\item Too few functions results in long and repetitive code
			\item Too many functions will result in sub-optimal performance and a code base that is very hard to read \begin{itemize}
				\item Every time a function is called a new frame needs to be pushed to the stack and we jump around the executable
			\end{itemize}
		\end{itemize}
	\end{itemize}
	\item How to define a function: \textbf{type name(arguments)}
	\begin{itemize}
		\item \textbf{type}: The return type of the function (can be \texttt{void})
		\item \textbf{name}: The function's name
		\item \textbf{arguments}: The input arguments to a function
		\begin{itemize}
			\item Specified as \textbf{type name} in a comma seperated list
			\item \textbf{Example}: \texttt{int add(int a, int b, int c) \{...\}}
		\end{itemize}
		\item Together, the function's name and arguments make up the signature
	\end{itemize}
	\item How to call a function: \texttt{name(arguments)}
	\begin{itemize}
	\item \textbf{Example}: \texttt{int sum = add(1, 2, 3); // sum = 6}
	\end{itemize}
\end{itemize}

\end{document}
